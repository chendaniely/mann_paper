\documentclass[10pt,letterpaper]{article}
\usepackage[latin1]{inputenc}
\usepackage{amsmath}
\usepackage{amsfonts}
\usepackage{amssymb}
\usepackage{graphicx}

\usepackage{caption}
\author{mark and daniel}
\title{mann paper}
\begin{document}

\section{Introduction}

\subsection{Watts}

The study of information cascades is about behavior (explicitly, states change).
Watts was fundamental to this approach...as was Centola and Barash, such that individuals use others actions as a driving input into their own actions.

Back to Watts, exemplifies some of the core notion here.
The route to cascades is complex precisely because the size of a shock alone does not predict cascade (in the sense that the system might receive many shocks of a small nature that seem equivalent, but in consequent are not w.r.t. a cascade.
Instead, it is the location of the shock, in the system w.r.t. the connectivity of and nature of agents. This theoretical work was motivated by clear examples of situations in which individual heuristics to rely on others behavior yielded clear benefit. 

Further work in this vein, what we call the threshold vein, further eludicated the importance of the effect of theoretical parameters on global outcomes (Centola; Barash).  The basic notion was that the specification of how the individual incorporates others' behavior matters.  E.g.,  Centola... PUT CHEN PARAGRPAH HERE.

CHEN PARAGRAPH BELOW\\
Binary decisions with externalities is a general class of problems
that can be used to model a wide variety of real-world problems.
It is a simple model that can be used to understand global information cascades
(e.g., fads, infectious diseases, collective action, innovations, computer virus,
and power grid failures).
Watts was able to show that three interacting features are needed for the model:
local dependencies ($k$), fractional thresholds ($\phi$), and heterogeneity.
He also defines the vulnerability of a node to how likely a node is able to carry or spread information.
Local dependencies refer to the network topology, where highly connected nodes are less vulnerable,
and fractional thresholds is a node parameter, where lower thresholds are more vulnerable.
Watts was able to show for information to globally cascade through a network (at least 10\%),
simply looking at either network topology or node-level attributes are not enough.
The interaction between these parameters drive global cascades,
and a critical boundary can be calculated to determine whether a global cascade will occur.

The interaction between local dependencies, fractional thresholds, and heterogeneity
can be illustrated with analogy.
Consider a population of people interacting with one another in a social network,
individuals in the network have varying number of friends (heterogeneous local dependencies).
Now, consider an infectious disease spreading in the network,
for a given virality of a disease, people will varying levels of susceptibility (heterogeneous fractional threshold).
Table \ref{table:k_vs_phi} shows the interaction between local dependencies of a network and
fractional thresholds of nodes.
In our example, if an individual needs a many exposes to a disease to be infected,
then it is unlikely the individual will contract the disease.
However, if an individual only needs to come into contact with a single infected person to be sick,
then it is more likely the individual will contract the disease.

Watts gave us insight into how we can look at networks of connected systems and
measure how information can flow in it.
However, the degree to which this applies to human systems is in question.


The limitation is that the Watts model is based on binary decisions.

Although behaviors can be measured in a binary manner (did the person perform a behavior, yes/no), mimetic mechanisms are only one of several types of information diffusion in humans (and non-human primates, some birds, some insects, etc.) 
That is, the process of performing a behavior is a complex process
that is not captured by using the Watts model.

\begin{table}[h]
\caption{Interaction between local dependency ($k$) and fractional thresholds ($\phi$)} 
\label{table:k_vs_phi} 
\begin{center}
\begin{tabular}{c|c|c|c|}
    & & local dependency & \\
    \hline 
    & & High & Low \\ 
    \hline 
    fractional threshold & High & Low & Low \\ 
    \hline 
    & Low & High & High \\ 
    \hline 
\end{tabular} 
\caption*{Table showing the interaction between local dependencies $k$ of a network,
          and fractional thresholds $\phi$ of nodes}
\end{center}
\end{table}

Watts gave us insight into how we can look at networks of connected systems and
measure how information can flow in it.
However, the degree to which this applies to human systems is in question.
The epidemic example above is an example of a human system,
but it does not explain how more complex social constructs or human behavior spread.
The limitation is that the Watts model is based on binary decisions.
Although behaviors can be measured in a binary manner (did the person perform a behavior, yes/no),
we as humans do not mimic people on a daily basis.
That is, the process of performing a behavior is a complex process
that is not captured by using the Watts model.

\hrulefill

``Social scientists are often interested in understanding how the dynamics
of social systems are driven by the behavior of individuals that make up
those systems. However, this process is hindered by the difficulty of
experimentally studying how individual behavioral tendencies lead to
collective social dynamics in large groups of people interacting over
time (Salganik 2009).''

physics

variation of network parameters

Watts varied 3 params

PH: contact networks and topology

Talk about watts major findings in the watts paper:
global cascades rely on:

local dependencies, fractional thresholds, and heterogeneity

\subsection{Sociology}

The sociology literature have advanced Watts' findings beyond understanding that diffusion among human systems may be dependent on thresholds, heterogeneity, and topology of the network.
For example, information diffusion may be modeled as a complex contagion,
where individuals require exposures from multiple sources,
rather than a single source (i.e., a simple contagion).
The rules for incorporating multiple sources rely on a threshold value
that is heterogeneous across the network (Chiang 2007, centola 2007, Morales 2014, Neill 2005).
Additionally, the pool of incoming sources need to be limited to immediate neighbors for
the diffusion of unpopular ideas and norms (Centola 2005 dilemma, Chiang 2007).

CHEN P BELOW\\
Sociology gives us one lens to the relationship between networks and human systems.
The sociology literature confirm many of Watts' findings:
diffusion among human systems is dependent on thresholds, heterogeneity, and topology of the network.
Information diffusion is commonly modeled as a complex contagion,

Topology of the network also plays a necessary role in contagion (Gibson 2005, Morales 2014).
For example, unpopular ideas are unable to spread in a fully connected network (centola 2005 dilemma).

The interaction between thresholds, heterogeneity, and topology govern whether a contagion
spreads within a network.
However, they can also create siloed communities or echo chambers
(Centola 2005, Chiang 2007, Centola 2015).
A balance between how similar or different members in a network are can govern how
information is spread in the network (Chiang 2007 Centola 2015)

The findings in sociology provide the bridge between theory and human systems.

\hrulefill

Topology of social networks are not given a priori (Centola 2015).
(want to fit this thought in somewhere... it's a good point and comment on how networks are studied)

within network diffusion

sociologist

variation to vertex params

simple to complex

look into centola citations

centola, watts $->$ sociology, galea needle sharing NYC

\subsection{Cog Neuro}

Although human behaviors can be seen as a binary process,
the decision making process is not a binary.
Beliefs, preferences, and constraints (BPC) models
can be used to model simple decisions from independent individuals (Bently 2007),
However, for more complex social settings, these models become less relevant.
The human decision making process is a complex system that is not fully understood.
Attempts to understand how structure relates to function ``model the propagation of
information as it relates to spatially constrained network properties (Gonzalex-Castillo 2015).''
We can use neural networks as a means to simulate human functions within a network of individuals.
This allows us to measure psychologically plausible decisions of individuals in a social context.


\hrulefill

scaling to networks and some of our current work

human information processing

\subsection{Why}

By incorporating the knowledge of simulations and network topologies from physics and mathematics
in a social context from sociology and information processing from cognitive neuroscience,
we can begin to understand the complex system of social diffusion of ideas and behaviors.
Identify echo chambers of ideas and how to break feedback loops
(Centola 2005, Chiang 2007, Centola 2015).

\hrulefill

why do we care about diffusion and global cascades?

effect our life

echo chambers?

\subsection{}

why we care?

why watts?

\section{Methods}

\section{Discussion}

Precice social engineering to improve population health.

\end{document}
